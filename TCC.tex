\documentclass[
    article,
    sumario=tradicional,
	% -- opções da classe memoir --
	12pt,				% tamanho da fonte
	openright,			% capítulos começam em pág ímpar (insere página vazia caso preciso)
	oneside,			% para impressão em verso e anverso. Oposto a oneside
	a4paper,			% tamanho do papel. 
	% -- opções da classe abntex2 --
	%chapter=TITLE,		% títulos de capítulos convertidos em letras maiúsculas
	section=TITLE,		% títulos de seções convertidos em letras maiúsculas
	%subsection=TITLE,	% títulos de subseções convertidos em letras maiúsculas
	%subsubsection=TITLE,% títulos de subsubseções convertidos em letras maiúsculas
	% -- opções do pacote babel --
	%english,			% idioma adicional para hifenização
	%french,				% idioma adicional para hifenização
	%spanish,			% idioma adicional para hifenização
	brazil				% o último idioma é o principal do documento
	]{abntex2}

\usepackage{ifthen,ifpdf}

\ifpdf
  \pdfpagewidth=\paperwidth
  \pdfpageheight=\paperheight
\fi

% --- 
% CONFIGURAÇÕES DE PACOTES
% --- 
% ---
% Pacotes básicos 
% ---
\usepackage{lmodern}			% Usa a fonte Latin Modern			
\usepackage[T1]{fontenc}		% Selecao de codigos de fonte.
\usepackage[utf8]{inputenc}		% Codificacao do documento (conversão automática dos acentos)
\usepackage{lastpage}			% Usado pela Ficha catalográfica
\usepackage{indentfirst}		% Indenta o primeiro parágrafo de cada seção.
\usepackage{color}				% Controle das cores
\usepackage{graphicx}			% Inclusão de gráficos
\usepackage{microtype} 			% para melhorias de justificação
\usepackage{ufc-abntex2}
\usepackage{enumitem}
\usepackage{amsmath}
\usepackage{booktabs}
\usepackage{multirow}
\usepackage{titlesec}
\usepackage{mathptmx}                               % Usa a fonte Times New Roman
\usepackage{float}
%formatação do Sumário
\usepackage{etoolbox}                               
% Usado para alterar a fonte da Section no Sumário
\usepackage[nogroupskip,nonumberlist,acronym]{glossaries}                %
% ---
		
% ---
% Pacotes adicionais, usados apenas no âmbito do Modelo Canônico do abnteX2
% ---
\usepackage{lipsum}				% para geração de dummy text
% ---

% ---
% Pacotes de citações
% ---
%\usepackage[brazilian,hyperpageref]{backref}	 % Paginas com as citações na bibl
%\usepackage[alf]{abntex2cite}	% Citações padrão ABNT

%Com et al nas referências
%\usepackage[alf, abnt-emphasize=bf, bibjustif, recuo=0cm, abnt-etal-cite=3, abnt-etal-text=it, abnt-etal-list=3]{abntex2cite} 

%Sem et al nas referências
\usepackage[alf, abnt-emphasize=bf, bibjustif, recuo=0cm, abnt-etal-cite=3, abnt-etal-text=it, abnt-etal-list=0]{abntex2cite} 

% Ambiente para alineas e e subalineas (incisos) com ponto
\newlist{alineascomponto}{itemize}{2}
\setlist[alineascomponto,1]{label={$\bullet$},topsep=0pt,itemsep=0pt,leftmargin=\parindent+\labelwidth-\labelsep}%
\setlist[alineascomponto,2]{label={--},topsep=0pt,itemsep=0pt,leftmargin=*}
\newlist{subalineascomponto}{enumerate}{1}
\setlist[subalineascomponto,1]{label={$\circ$},topsep=0pt,itemsep=0pt,leftmargin=*}%
% ---

% Ambiente para alineas e e subalineas (incisos) com numeros
\newlist{alineascomnumero}{enumerate}{2}
\setlist[alineascomnumero,1]{label={$\arabic*$.},topsep=0pt,itemsep=0pt,leftmargin=\parindent+\labelwidth-\labelsep}%
\setlist[alineascomnumero,2]{label={--},topsep=0pt,itemsep=0pt,leftmargin=*}
\newlist{subalineascomnumero}{enumerate}{1}
\setlist[subalineascomnumero,1]{label={$\arabic*$.},topsep=0pt,itemsep=0pt,leftmargin=*}%
% ---

% ---
% Configurações do pacote backref
% Usado sem a opção hyperpageref de backref
%\renewcommand{\backrefpagesname}{Citado na(s) página(s):~}
% Texto padrão antes do número das páginas
%\renewcommand{\backref}{}
% Define os textos da citação
%\renewcommand*{\backrefalt}[4]{
%	\ifcase #1 %
%		Nenhuma citação no texto.%
%	\or
%		Citado na página #2.%
%	\else
%		Citado #1 vezes nas páginas #2.%
%	\fi}%
% ---

% ---
% Configurações de aparência do PDF final

% alterando o aspecto da cor azul
\definecolor{blue}{RGB}{41,5,195}

% informações do PDF
\makeatletter
\hypersetup{
     	%pagebackref=true,
		pdftitle={\@title}, 
		pdfauthor={\@author},
    	pdfsubject={\imprimirpreambulo},
	    pdfcreator={LaTeX with abnTeX2},
		pdfkeywords={abnt}{latex}{abntex}{abntex2}{trabalho acadêmico}, 
		colorlinks=true,       		% false: boxed links; true: colored links
    	linkcolor=blue,          	% color of internal links
    	citecolor=blue,        		% color of links to bibliography
    	filecolor=magenta,      		% color of file links
		urlcolor=blue,
		bookmarksdepth=4
}
\makeatother
% --- 

\makeatletter
\def\@biblabel#1{}
\renewenvironment{thebibliography}[1]
     {\section*{\refname}%
      \@mkboth{\MakeUppercase\refname}{\MakeUppercase\refname}%
      \list{\@biblabel{\@arabic\c@enumiv}}%
           {\settowidth\labelwidth{\@biblabel{#1}}%
            \leftmargin\labelwidth
            %\advance\leftmargin\labelsep
            \@openbib@code
            \usecounter{enumiv}%
            \let\p@enumiv\@empty
            \renewcommand\theenumiv{\@arabic\c@enumiv}}%
      \sloppy
      \clubpenalty4000
      \@clubpenalty \clubpenalty
      \widowpenalty4000%
      \sfcode`\.\@m}
     {\def\@noitemerr
       {\@latex@warning{Empty `thebibliography' environment}}%
      \endlist}
\makeatother

% --- 
% Espaçamentos entre linhas e parágrafos 
% --- 

% O tamanho do parágrafo é dado por:
\setlength{\parindent}{1.5cm}


\setlength{\parindent}{2cm}

% ---
% compila o indice
% ---
\makeindex
% ---


% Define as margens do documento
\setlrmarginsandblock{3cm}{2cm}{*} % externa / interna
\setulmarginsandblock{3cm}{2cm}{*} % superior / inferior

% Define o espaço entre linhas para 1.5 cm
%\OnehalfSpacing	Bug no ABNTEX2?
\renewcommand{\baselinestretch}{1.5}



\usepackage{hyperref}% http://ctan.org/pkg/hyperref
\hypersetup{%
  colorlinks = false,
  linkcolor  = black
}

% Informações de dados para CAPA e FOLHA DE ROSTO
\titulo{TÍTULO TÍTULO TÍTULO TÍTULO  TÍTULO TÍTULO TÍTULO TÍTULO TÍTULO TÍTULO TÍTULO TÍTULO TÍTULO TÍTULO }
\autor{Nome Autor}
\local{Quixadá}
\data{Abril, 2016}
\orientador{Nome Orientador}
\coorientador{Nome Coorientador}

% Escolher curso: Redes de Computadores (rc), Eng.Software (es), Ciências da Computação (cc) ou Sist.Informação (si)
%\instituicao{%
Universidade Federal do Ceará \par
Campus Quixadá \par
Curso de Redes de Computadores
}
\tipotrabalho{Trabalho de Conclusão de Curso (Monografia)}
\preambulo{Monografia apresentada ao Curso de Redes de Computadores do Campus Quixadá da Universidade Federal do Ceará, como requisito parcial para obtenção do Título de Tecnólogo em Redes de Computadores.}

%\instituicao{%
Universidade Federal do Ceará \par
Campus Quixadá \par
Curso de Sistemas de Informação
}
\tipotrabalho{Trabalho de Conclusão de Curso (Monografia)}
\preambulo{Monografia apresentada ao Curso de Sistemas de Informação do Campus Quixadá da Universidade Federal do Ceará, como requisito parcial para obtenção do Título de Bacharel em Sistemas de Informação.}

\instituicao{%
Universidade Federal do Ceará \par
Campus de Quixadá \par
Curso de Engenharia de Software
}
\tipotrabalho{Trabalho de Conclusão de Curso (Monografia)}
\preambulo{Monografia apresentada ao Curso de Engenharia de Software do Campus Quixadá da Universidade Federal do Ceará, como requisito parcial para obtenção do Título de Bacharel em Engenharia de Software.}

%\instituicao{%
Universidade Federal do Ceará \par
Campus Quixadá \par
Curso de Ciências da Computação
}
\tipotrabalho{Trabalho de Conclusão de Curso (Monografia)}
\preambulo{Monografia apresentada ao Curso de Ciências da Computação do Campus Quixadá da Universidade Federal do Ceará, como requisito parcial para obtenção do Título de Bacharel em Ciências da Computação.}
 

%%criar um novo estilo de cabeçalhos e rodapés
\makepagestyle{header_style}
  %%cabeçalhos
  \makeoddhead{header_style} %%pagina ímpar ou com oneside
     {}
     {}
     {\thepage} 


\begin{document}
\frenchspacing 

%Formatação de título de seções
\titleformat{\section}
{\normalfont\normalsize\bfseries}{\thesection}{1em}{}
\titleformat{\subsection}
{\normalfont\normalsize\bfseries}{\thesubsection}{1em}{}
\titleformat{\subsubsection}
{\normalfont\normalsize\bfseries}{\thesubsubsection}{1em}{}

\renewcommand{\cftsectionfont}{\normalfont\normalsize}   
\renewcommand{\cftsubsectionfont}{\normalfont\normalsize} 
\renewcommand{\cftsubsubsectionfont}{\normalfont\normalsize}  

% ----------------------------------------------------------
% ELEMENTOS PRÉ-TEXTUAIS
% ----------------------------------------------------------
% \pretextual
% Capa
\imprimircapa

%----------- Apenas TCC 2
% Folha de rosto (* indica que haverá a ficha bibliográfica)
%\imprimirfolhaderosto

% Ficha Bibliográfica
%\include{fixos/fichabibliografica}

% Errata
%\include{editaveis/errata}

% Folha de Aprovação
% DEVE ser modificada para adicionar os membros da banca
%% ---
% Inserir folha de aprovação
% ---

% Isto é um exemplo de Folha de aprovação, elemento obrigatório da NBR
% 14724/2011 (seção 4.2.1.3). Você pode utilizar este modelo até a aprovação
% do trabalho. Após isso, substitua todo o conteúdo deste arquivo por uma
% imagem da página assinada pela banca com o comando abaixo:
%
% \includepdf{folhadeaprovacao_final.pdf}
%
\begin{folhadeaprovacao}

  \begin{center}
    {\bfseries\Large\imprimirautor}
    \vspace{1cm}

    \begin{center}
      \bfseries\Large\imprimirtitulo
    \end{center}

    \vspace{2cm}
    \begin{minipage}{\textwidth}
        \imprimirpreambulo
        \\ \\ \\
        Aprovada em: \_\_/\_\_/\_\_\_\_
    \end{minipage}%
     
    \vspace{2cm}
	\textbf{BANCA EXAMINADORA}
   \end{center}
	

   \assinatura{\imprimirorientador \space (Orientador) \\ Universidade Federal do Ceará (UFC)}
   \assinatura{\imprimircoorientador \space (Co-Orientador) \\ Universidade Federal do Ceará (UFC)}
   %DEFINA AQUI OS DEMAIS MEMBROS DA BANCA
   \assinatura{Prof. Msc. Da Silva \\ Universidade Federal do Ceará (UFC)}
   %\assinatura{Prof. Dr. Alguma Coisa \\ Instituição}
   %\assinatura{Prof. Msc. Alguma Coisa \\ Instituição}
      
%   \begin{center}
%    \vspace*{0.5cm}
%    {\large\imprimirlocal}
%    \par
%    {\large\imprimirdata}
%    \vspace*{1cm}
%  \end{center}
  
\end{folhadeaprovacao}
% ---

%\imprimirfolhadeaprovacao

% Dedicatória
%% ---
% Dedicatória
% ---
\begin{dedicatoria}
   \vspace*{\fill}
   	\begin{flushright}
   \noindent
    Este trabalho é dedicado às crianças adultas que, quando pequenas, sonharam em se tornar cientistas.
   	\end{flushright}
\end{dedicatoria}
% ---

% Agradecimentos
%\include{editaveis/agradecimentos}

% Epígrafe
%\include{editaveis/epigrafe}

% RESUMOS
%\include{resumo/ptbr}
%\include{resumo/us}
%% resumo em francês 
\begin{resumo}[Résumé]
 \begin{otherlanguage*}{french}
    Il s'agit d'un résumé en français.
 
   \textbf{Mots-clés}: latex. abntex. publication de textes.
 \end{otherlanguage*}
\end{resumo}

%% resumo em espanhol
\begin{resumo}[Resumen]
 \begin{otherlanguage*}{spanish}
   Este es el resumen en español.
  
   \textbf{Palabras clave}: latex. abntex. publicación de textos.
 \end{otherlanguage*}
\end{resumo}
% ---

% Lista de ilustrações
%\pdfbookmark[0]{\listfigurename}{lof}
%\listoffigures*
%\clearpage

% Lista de tabelas
%\pdfbookmark[0]{\listtablename}{lot}
%\listoftables*
%\cleardoublepage

% Abreviaturas e Siglas
%\include{editaveis/siglas}

% Símbolos
%\include{editaveis/simbolos}
%----------- fim - Apenas TCC 2


% Sumário

\imprimirsumario

% ----------------------------------------------------------
% ELEMENTOS TEXTUAIS
% ----------------------------------------------------------
\textual

%aplicação de estilo de cabeçalho
\pagestyle{header_style}

%uso do input pois o include dá quebra de página no final
\section{INTRODUÇÃO}

Inicia-se contextualizando o tema do trabalho e considerando os seguintes aspectos no desenvolvimento da introdução: \cite{teste}

\begin{alineascomponto}
    \item O que o projeto enfoca? \textbf{Problema}(s) a solucionar ou equacionar, com informações sobre ele(s).
    \item O projeto atende a quem? \textbf{Público-alvo} a ser beneficiado com a ação. Deve-se descrever as características socioeconômicas, educacionais, culturais e outras que se julgar importante do público-alvo.
    \item Justificativa no presente – o projeto existe por quê? Qual a \textbf{relevância} do projeto? qual a influência que a ação proposta no projeto pode exercer na vida do público-alvo?
    \item Em alguns trabalhos, expõe-se as consequências no médio/longo prazo –  o projeto contribui para quê? Impacto do projeto as transformações positivas e duradouras esperadas.
\end{alineascomponto}

A introdução deve necessariamente contextualizar o trabalho no conhecimento atual do seu tema. Assim, deve-se citar brevemente o que outras pessoas tem feito de similar ao  trabalho proposto, acrescentando suas similaridades e diferenças com elas. Essa apresentação nesta seção da introdução é breve o suficiente para justificar a existência do seu trabalho, respondendo: de que forma ele se diferencia do que já existe? Apresentação detalhada é feita na seção “2. Trabalhos Relacionados”.

Todo o texto deve ser escrito no modo impessoal.

Quanto à formatação do texto, deve-se observar que a numeração de páginas começa a contar após a capa, e começa a ser exibida apenas na introdução.

\section{TRABALHOS RELACIONADOS}

No cotidiano, um bom ponto de partida para se resolver um problema é procurar soluções já existentes para utilizá-las. Costumeiramente, as soluções que já existentes não se aplicam diretamente ao nosso caso, precisando ser adaptadas.

Assim, antes de se começar a resolver questões de pesquisa, é preciso conhecer o que tem de mais atual no seu tema.  Usando a abordagem de \citeonline{wazlawick2014metodologia} para explicar a necessidade de se conhecer trabalhos relacionados, cabe lembrar que antes de se construir uma nova ponte é importante conhecer os tipos de pontes que já existem; é preciso conhecer qual a atualidade do assunto estudado. Do contrário, pode estar construindo uma catapulta achando que se trata da melhor forma de atravessar um rio!

Para cada texto relacionado relevante encontrado, escreva: 1) qual a relação dele com seu trabalho, de que forma contribui; 2) que maneira a proposta se assemelha ao trabalho relacionado, ou seja, qual a relação direta entre os dois; 3) por fim, informa-se em que aspecto a proposta se difere do trabalho relacionado. Escreva de forma fluente, de maneira que não se perceba três fragmentos no texto.

A extensão e a profundidade necessária deste levantamento de trabalhos relacionados são determinados pelo perfil de sua área de conhecimento, e pelo seu orientador. Mas uma coisa é certa: não se pode dizer que seu trabalho é bom e justificável, se não houver como compará-lo a outros trabalhos que já existem.

\begin{alineas}

\item[a.] Corpo de Conhecimento: quando dela se utiliza conceitos já estabelecidos; este conteúdo que aparece mais destacadamente na seção Referencial teórico/revisão bibliográfica do seu trabalho;

\item[b.] Metodologia: alguns trabalhos são uma boa referência para o estabelecimento da metodologia de pesquisa; este conteúdo em geral subsidia a seção Procedimentos Metodológicos.

\item[c.] Trabalho relacionado: trabalhos que possuam mesma motivação, objetivo ou, em alguns casos específicos, metodologia. Ao se ler um bom trabalho relacionado, automaticamente surgem pensamentos como “ah, ele fez assim e posso fazer parecido” ou “não! esse aspecto do trabalho poderia ser melhor, prefiro fazer assim e assim”.  Se esses tipos de pensamento surgirem, então terá encontrado um bom texto candidato a ser considerado Trabalho Relacionado.

\end{alineas}

Algumas referências podem facilitar muito a sua busca por conhecer a atualidade do tema de estudo proposto, ajudando o pesquisador em diferentes aspectos do seu trabalho. Tipicamente, estes são os materiais denominados surveys (levantamentos), podendo ser compilações de:

\begin{alineas}

\item[d.] Estado-da-arte: artigos que apresentem conceitos mais recentes, estabelecidos na literatura científica;

\item[e.] Estado-da-prática: semelhante ao anterior, mas com foco no que está estabelecido atualmente como status quo da prática profissional.

\end{alineas}

Uma coisa é certa: enquanto o pesquisador não encontrar trabalhos relacionados à sua proposta, pode ter a certeza de que não procurou corretamente!

\section{OBJETIVOS}

\subsection{Objetivo Geral}

O objetivo deve ser apresentado na forma de um único parágrafo, tendo como elemento central um único verbo de ação expressando o que será realizado. O que será o produto final? Onde se aplica? O Objetivo Geral deve ser claro, mensurável, realista, atingível em um determinado tempo.

\subsection{Objetivos específicos}

\begin{alineas}
    
    \item Devem estar vinculados ao objetivo geral e são produtos intermediários, que deverão ser cumpridas ao longo da pesquisa.
    \item Os objetivos específicos também devem ser mensuráveis, viáveis em um determinado tempo e relacionados às necessidades.
    \item Tipicamente, um projeto possui três objetivos de pesquisa. Sugere-se começar definindo três deles, e ajustando conforme a natureza do trabalho. 

\end{alineas}

\section{FUNDAMENTAÇÃO TEÓRICA}

De uma maneira simplificada, teoria é aquilo que explica porque algo é como é. Esta seção, deve descrever os conceitos necessários para explicar as decisões a serem tomadas no desenvolver da pesquisa.

Antes de se iniciar as subseções é preciso fazer uma breve apresentação das subseções seguintes, com um bom encadeamento lógico relacionando-as.

Usar diferentes seções para diferentes conceitos-chave do trabalho. Um ponto de partida é considerar três conceitos chave extraídos do título do trabalho.

\subsection{Conceito chave-1}

O que é manga? Fruta, parte da roupa, ou um verbo?

Em cada subseção, é preciso informar ao leitor qual o significado adotado para cada conceito utilizado na pesquisa. Conceitue ou descreva cada um deles. Caso existam diferentes abordagens para um mesmo conceito, deixe claro qual aquela que será adotada.

A fundamentação teórica/revisão bibliográfica não é uma lista de verbetes com explicações. Não basta dizer o que é cada peça usada na montagem do trabalho; tem-se que explicar a função de cada uma e como ela interage com as outras peças. Ao final de cada seção, é preciso informar ao leitor a relação daquele conceito com o trabalho. 

\subsection{Conceito chave-2}

Convém ser caridoso com o leitor: usar uma escrita didática, com boas explicações; o leitor merece reconhecimento por se dispor a conhecer o trabalho, além do fato de que nem sempre entende bem do conteúdo lido.  Revisar, revisar, revisar, pelo menos três vezes, nunca é demais. Evita-se resumir capítulos de livros: uma boa fundamentação apresenta os conceitos relevantes para o trabalho e faz as conexões entre eles. 

O conteúdo de sites como Wikipédia e blogs não são reconhecidos como cientificamente válidos porque seu conteúdo nem sempre é confiável. Usa-se anais de eventos, bons livros, periódicos, bancos de teses e dissertações. Para buscas na internet, sugere-se usar o buscador Google Acadêmico, indexadores como Scielo e BDBComp, o Portal de Periódicos Capes.

Uma forma prática de encontrar os primeiros materiais é procurar nos anais de importantes conferências da sua área de estudo, ou em periódicos relacionados. É comum se precisar de ajuda do orientador para definir quais os principais eventos e periódicos tratam do tema de estudo.

\subsection{Conceito chave-3}

Um texto pode conter diferentes tipos de ilustração, que são: uma "designação genérica de imagem que ilustra ou elucida um texto. São consideradas ilustrações: desenho, esquemas, fluxograma, gráfico, mapa, organograma, planta, quadro, retrato, figura, imagem, entre outros" ~ \cite{ufc_guia}. Todos eles podem ser rotulados pela palavra "Figura", como na Figura 1, ou receber denominações específicas como no Quadro 1. 
Usa-se a denominação Tabelas, que tem formatação específica, apenas em caso de dados numéricos. Quando se tratar de dados textuais, deve-se denominar Quadro.

%% Caso a imagem seja pequena, coloque borda para preencher a largura da página
\begin{figure}[htbp]
	\centering	
	\UFCfig{
	\Caption{Exemplo de figura \label{graf2}}		
	}{
	    \includegraphics[scale=0.4]{figuras/figura_1} % altere o atributo scale para o tamanho da figura
	}{
	\Fonte{Próprio Autor.}
    }
\end{figure}

\begin{table}[h!]
	\IBGEtab{
	\Caption{\label{tabela-ibge} Um Exemplo de tabela alinhada que pode ser longa ou curta, conforme padrão IBGE.}%
	}{%
		\begin{tabular}{cccccc}
			\toprule
			Nome & Nascimento & Documento &  Nascimento & Documento & Nascimento \\
			\midrule \midrule
			Maria da Silva & 11/11/1111 & 111.111.111-11 & 111.111.111-11 & 111.111.111-11 & 111.111.111-11 \\
			Maria da Silva & 11/11/1111 & 111.111.111-11 & 111.111.111-11 & 111.111.111-11 & 111.111.111-11 \\
			Maria da Silva & 11/11/1111 & 111.111.111-11 & 111.111.111-11 & 111.111.111-11 & 111.111.111-11 \\
			\bottomrule
		\end{tabular}%
	}{%
	\Fonte{Produzido pelos autores}
}
\end{table}


\begin{quadro}[h!]	
	\centering
	\UFCqua{
	    \Caption{\label{qua:exemplo-1} Exemplo de Quadro}		
	    }{
		\begin{tabular}{|c|c|c|}
			\hline
			Quisque & faucibus & pharetra \\
			\hline
			E1 & F1 & Complete coverage by a single transcript \\
			\hline
			E2 & F1 & Complete coverage by more than \\
			\hline
			E3 & F1 & Partial coverage \\
			\hline
			E4 & F1 & Partial coverage \\
			\hline
			E5 & F1 & Partial coverage \\
			\hline
			E6 & F1 & Partial coverage \\
			\hline
			E7 & F1 & Partial coverage \\
			\hline
		\end{tabular}
	}{
		\Fonte{Elaborado pelo autor}
	}
\end{quadro}
	

\section{PROCEDIMENTOS METODOLÓGICOS}

Procedimentos Metodológicos relaciona-se ao passo-a-passo da execução do trabalho pesquisa: como se obterá os dados necessários para respondem à sua questão de pesquisa? Um bom ponto de partida para escrever os procedimentos é detalhar extensamente cada objetivo específico.

Para que os resultados encontrados sejam considerados válidos, é preciso respeitar e seguir as tradições de cada área de pesquisa. As estratégias de levantamento de dados, e de registro e análise do material coletado, mudam conforme a natureza da pesquisa. Seguem alguns exemplos:


\begin{alineascomponto}
    \item Experimentos, o que inclui desenvolvimento de protótipos ou produtos;
    \item Análise de documentos;
    \item Entrevistas, em duas diversas variações;
    \item Observações, em suas diversas variações;
    \item Metodologia de desenvolvimento de software considerada;
    \item Características das ferramentas a serem utilizadas, e demais recursos necessários;
    \item Para cada uma das estratégias exemplificadas, deve-se responder: 
    \item O que? (atividade)
    \item Como? (técnica)
    \item Quando? (período), podendo ser apresentado apenas no cronograma.
    \item Campo da pesquisa e a amostra de dados a ser considerada (quando aplicável)

\end{alineascomponto}

Quando se tratar de desenvolvimento de ferramenta, vários dos itens anteriormente sugeridos serão substituídos pelo método de desenvolvimento utilizado.

\subsection{Subseção 1}

Se necessário, detalha-se as etapas em subseções. Na primeira versão do projeto, sugere-se usar generosa quantidade de subseções, mesmo que elas fiquem com pouco conteúdo no início. Em versões mais amadurecidas, pode-se unir subseções em grupos, desde que a apresentação do conteúdo de cada uma delas já esteja saturada.
O primeiro passo dos procedimentos não deve ser “revisão bibliográfica” ou “estudar tal e tal conceito”. Conforme \citeonline{wazlawick2014metodologia}, estudar é obrigação do pesquisador, e não uma etapa da pesquisa.

Aceita-se revisão bibliográfica como primeiro passo apenas em casos muito específicos, em uma área do conhecimento muito nova, e que ainda não se tem o conhecimento já desenvolvido e publicado. 

\subsection{Subseção 2}

Ao escrever o passo sobre “análise” ou “avaliação”, é imprescindível informar quais os critérios de análise. Tais critérios, já estarão detalhados na seção de fundamentação teórica, e serão apenas citados nesta seção de procedimentos metodológicos.

\subsection{Cronograma de Execução}

\textit{A última seção dos procedimentos é o cronograma. Apresente a versão que entregará à banca ao final do semestre. Se seus procedimentos não estiverem organizados em subseções, esta será a subseção 4.1. Após ler, remova este texto explicativo.}

\begin{table}[H]
\centering
\resizebox{\textwidth}{!}{\begin{tabular}{|l|c|c|c|c|c|c|c|c|c|c|c|c|c|c|}
\hline
\multicolumn{1}{|c|}{\multirow{2}{*}{ATIVIDADES}} & \multicolumn{14}{c|}{2015} \\ \cline{2-15} 
\multicolumn{1}{|c|}{} & \multicolumn{2}{c|}{Mai} & \multicolumn{2}{c|}{Jun} & \multicolumn{2}{c|}{Jul} & \multicolumn{2}{c|}{Ago} & \multicolumn{2}{c|}{Set} & \multicolumn{2}{c|}{Out} & \multicolumn{2}{c|}{Nov} \\ \hline
\begin{tabular}[c]{@{}l@{}}(escreva aqui a primeira etapa DA EXECUÇÃO,\\  prevista para antes do término do TCC1)\end{tabular} & x & x & x &  &  &  &  &  &  &  &  &  &  & - \\ \hline
Defesa do projeto &  &  &  & x &  &  &  &  &  &  &  &  &  & - \\ \hline
(descreva aqui a segunda etapa da execução) &  &  &  &  & x & x &  &  &  &  &  &  &  & - \\ \hline
(descreva aqui a terceira etapa da execução) &  &  &  &  &  & x & x &  &  &  &  &  &  & - \\ \hline
(descreva aqui a quarta etapa da execução) &  &  &  &  &  &  & x & x &  &  &  &  &  & - \\ \hline
......inclua mais linhas se necessário &  &  &  &  &  &  &  & x &  &  &  &  &  & - \\ \hline
(Execução/coleta de dados de ...) &  &  &  &  &  &  &  & x & x & x &  &  &  & - \\ \hline
(Análise dos Dados) &  &  &  &  &  &  &  &  &  & x & x &  &  & - \\ \hline
(Avaliação da Execução) &  &  &  &  &  &  &  &  &  &  & x & x &  & - \\ \hline
Revisão final da monografia &  &  &  &  &  &  &  &  &  &  &  & x & x & - \\ \hline
Defesa do Trabalho Final &  &  &  &  &  &  &  &  &  &  &  &  & x & - \\ \hline
\end{tabular}}
\end{table}

\section{RESULTADOS PRELIMINARES}

Esta seção estará em vazia na primeira versão de projeto a ser entregue na disciplina Projeto de Pesquisa. Deve ficar vazia mesmo, não sendo excluída.

Após esta entrega, está previsto que a pesquisa seja iniciada e este é o local reservado para que se incluam resultados parciais antes da defesa. Para a defesa, deve-se combinar previamente com o orientador se haverá material suficiente que justifique manter esta seção. Em caso positivo, ela conterá o relato do andamento do trabalho a ser apresentado para a banca avaliadora. Em caso negativo, esta seção é excluída.

% ----------------------------------------------------------
% ELEMENTOS PÓS-TEXTUAIS
% ----------------------------------------------------------
%\postextual

% Referências bibliográficas

\renewcommand{\refname}{REFERÊNCIAS}
\begingroup
\raggedright
\bibliography{bibtex/referencias}
\endgroup
\addcontentsline{toc}{section}{REFERÊNCIAS}
      
% Glossário (Consulte o manual da classe abntex2 para orientações sobre o glossário)
%\glossary

% Apêndices
\apendice{APÊNDICE A}
\addcontentsline{toc}{section}{APÊNDICE A}

Contém materiais de leitura opcional e complementar produzidos pelo autor da pesquisa, incluindo os instrumentos de coleta de dados a serem utilizados. Se não for utilizada, esta seção deve ser removida já na versão 1 do projeto.

% Anexos
% ----------------------------------------------------------
% Anexos
% ---
% Inicia os anexos
% ---

\apendice{ANEXO A}
\addcontentsline{toc}{section}{ANEXO A}

Contém documentos de outros autores, quando aplicável. Se não for utilizada, esta seção deve ser removida já na versão 1 do projeto.

%---------------------------------------------------------------------
% INDICE REMISSIVO
%---------------------------------------------------------------------
%\phantompart
%\printindex
%---------------------------------------------------------------------

\end{document}