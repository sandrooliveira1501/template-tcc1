\section{PROCEDIMENTOS METODOLÓGICOS}

Procedimentos Metodológicos relaciona-se ao passo-a-passo da execução do trabalho pesquisa: como se obterá os dados necessários para respondem à sua questão de pesquisa? Um bom ponto de partida para escrever os procedimentos é detalhar extensamente cada objetivo específico.

Para que os resultados encontrados sejam considerados válidos, é preciso respeitar e seguir as tradições de cada área de pesquisa. As estratégias de levantamento de dados, e de registro e análise do material coletado, mudam conforme a natureza da pesquisa. Seguem alguns exemplos:


\begin{alineascomponto}
    \item Experimentos, o que inclui desenvolvimento de protótipos ou produtos;
    \item Análise de documentos;
    \item Entrevistas, em duas diversas variações;
    \item Observações, em suas diversas variações;
    \item Metodologia de desenvolvimento de software considerada;
    \item Características das ferramentas a serem utilizadas, e demais recursos necessários;
    \item Para cada uma das estratégias exemplificadas, deve-se responder: 
    \item O que? (atividade)
    \item Como? (técnica)
    \item Quando? (período), podendo ser apresentado apenas no cronograma.
    \item Campo da pesquisa e a amostra de dados a ser considerada (quando aplicável)

\end{alineascomponto}

Quando se tratar de desenvolvimento de ferramenta, vários dos itens anteriormente sugeridos serão substituídos pelo método de desenvolvimento utilizado.

\subsection{Subseção 1}

Se necessário, detalha-se as etapas em subseções. Na primeira versão do projeto, sugere-se usar generosa quantidade de subseções, mesmo que elas fiquem com pouco conteúdo no início. Em versões mais amadurecidas, pode-se unir subseções em grupos, desde que a apresentação do conteúdo de cada uma delas já esteja saturada.
O primeiro passo dos procedimentos não deve ser “revisão bibliográfica” ou “estudar tal e tal conceito”. Conforme \citeonline{wazlawick2014metodologia}, estudar é obrigação do pesquisador, e não uma etapa da pesquisa.

Aceita-se revisão bibliográfica como primeiro passo apenas em casos muito específicos, em uma área do conhecimento muito nova, e que ainda não se tem o conhecimento já desenvolvido e publicado. 

\subsection{Subseção 2}

Ao escrever o passo sobre “análise” ou “avaliação”, é imprescindível informar quais os critérios de análise. Tais critérios, já estarão detalhados na seção de fundamentação teórica, e serão apenas citados nesta seção de procedimentos metodológicos.

\subsection{Cronograma de Execução}

\textit{A última seção dos procedimentos é o cronograma. Apresente a versão que entregará à banca ao final do semestre. Se seus procedimentos não estiverem organizados em subseções, esta será a subseção 4.1. Após ler, remova este texto explicativo.}

\begin{table}[H]
\centering
\resizebox{\textwidth}{!}{\begin{tabular}{|l|c|c|c|c|c|c|c|c|c|c|c|c|c|c|}
\hline
\multicolumn{1}{|c|}{\multirow{2}{*}{ATIVIDADES}} & \multicolumn{14}{c|}{2015} \\ \cline{2-15} 
\multicolumn{1}{|c|}{} & \multicolumn{2}{c|}{Mai} & \multicolumn{2}{c|}{Jun} & \multicolumn{2}{c|}{Jul} & \multicolumn{2}{c|}{Ago} & \multicolumn{2}{c|}{Set} & \multicolumn{2}{c|}{Out} & \multicolumn{2}{c|}{Nov} \\ \hline
\begin{tabular}[c]{@{}l@{}}(escreva aqui a primeira etapa DA EXECUÇÃO,\\  prevista para antes do término do TCC1)\end{tabular} & x & x & x &  &  &  &  &  &  &  &  &  &  & - \\ \hline
Defesa do projeto &  &  &  & x &  &  &  &  &  &  &  &  &  & - \\ \hline
(descreva aqui a segunda etapa da execução) &  &  &  &  & x & x &  &  &  &  &  &  &  & - \\ \hline
(descreva aqui a terceira etapa da execução) &  &  &  &  &  & x & x &  &  &  &  &  &  & - \\ \hline
(descreva aqui a quarta etapa da execução) &  &  &  &  &  &  & x & x &  &  &  &  &  & - \\ \hline
......inclua mais linhas se necessário &  &  &  &  &  &  &  & x &  &  &  &  &  & - \\ \hline
(Execução/coleta de dados de ...) &  &  &  &  &  &  &  & x & x & x &  &  &  & - \\ \hline
(Análise dos Dados) &  &  &  &  &  &  &  &  &  & x & x &  &  & - \\ \hline
(Avaliação da Execução) &  &  &  &  &  &  &  &  &  &  & x & x &  & - \\ \hline
Revisão final da monografia &  &  &  &  &  &  &  &  &  &  &  & x & x & - \\ \hline
Defesa do Trabalho Final &  &  &  &  &  &  &  &  &  &  &  &  & x & - \\ \hline
\end{tabular}}
\end{table}